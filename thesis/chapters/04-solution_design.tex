\chapter{Design and Implementation of the Solution}
\label{cha:solution_design}

\section{System Architecture}
% Provide a high-level diagram of your setup.
% e.g., Event source -> Message Queue -> KEDA -> Application Pods

\section{Tools and Technologies}
% List the specific tools you used.
% - Kubernetes distribution (e.g., Minikube, Docker Desktop, GKE)
% - Message Broker (e.g., RabbitMQ)
% - Application language/framework (e.g., Python, Go, Node.js)
% - KEDA version

\section{Test Application}
\subsection{Application Logic}
% Describe what your sample application does.
% e.g., a worker that processes messages from a queue.
\subsection{Dockerfile}
% Include and explain the Dockerfile for your application.

\section{Environment Setup}
\subsection{Kubernetes Cluster Setup}
% How did you set up the cluster?
\subsection{Deploying the Event Source}
% e.g., RabbitMQ deployment YAML.
\subsection{Deploying the Application}
% Deployment YAML for your application.

\section{Configuring KEDA for Autoscaling}
\subsection{The \texttt{ScaledObject} Configuration}
% Show and explain your ScaledObject YAML.
% Discuss the key parameters: polling interval, cooldown period, triggers, etc.
\subsection{Verifying the Setup}
% How did you check that KEDA was working? `kubectl get hpa`, etc. 