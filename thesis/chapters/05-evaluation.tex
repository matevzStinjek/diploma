\chapter{Evaluation and Results}
\label{cha:evaluation}

\section{Test Environment}
% Reiterate the hardware and software specifications.
% This is important for reproducibility.

\section{Evaluation Metrics}
% What are you measuring?
% - Scaling time (from event to new pod running)
% - Resource utilization (CPU/memory of pods)
% - Queue length or other trigger metric
% - Cost savings (if applicable)

\section{Test Scenarios}
% Describe the experiments you ran.
\subsection{Scenario 1: Sudden Burst of Traffic}
% What happens when a large number of events arrive at once?
\subsection{Scenario 2: Gradual Increase in Traffic}
% How does the system scale with a slow, steady increase in load?
\subsection{Scenario 3: Scaling to Zero}
% Demonstrate and measure the scale-down behavior when there are no events.

\section{Presenting the Results}
% Use graphs and tables to show your data.
% For each scenario, present the metrics you defined.
% Example: A graph showing queue length vs. number of pods over time.

\section{Analysis of Results}
% What do the results mean?
% Did KEDA perform as expected?
% What are the trade-offs you observed?
% Compare the results between different scenarios. 